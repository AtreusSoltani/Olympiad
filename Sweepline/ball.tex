\def\problemCode{\LR{Ball}}
\def\problemFarsiTitle{توپ}

\begin{problem}

  سوال
  \href{https://codeforces.com/problemset/problem/12/D}{\LR{Codefoces 12D}}
  را می‌خواهیم با همین روش حل کنیم.
  
  در این مسئله 
  $n$
  نفر به یک مهمانی دعوت شده‌اند. هر فرد
  $3$
  مشخصه
  زیبایی،
  هوش و
  دارایی دارد که آن‌ها را به‌ترتیب با
  $B_i$
  ،
  $I_i$
  و
  $R_i$
  نشان می‌دهیم. فرد 
  $i$
  بعد از مهمانی افسردگی می‌گیرد اگر فردی
  ($j$)
  در مهمانی باشد که 
  $B_j > B_i$
  ،
  $I_j > I_i$
  و
  $R_j > R_i$
  باشد. تعداد افرادی که بعد از مهمانی افسردگی می‌گیرند را پیدا کنید.

  این بار نقاط در فضای
  $3$
  بعدی هستند. با استفاده از 
  \LR{Sweepline}
  و
  روشن-خاموش بودن نقطه‌ها یکی از ابعاد را کاهش می‌دهیم.
  فرض کنید افراد را بر اساس
  $B$
  به صورت نزولی مرتب می‌کنیم.
  برای سادگی فرض کنید که هیچ دو نفری 
  $B$
  یکسان ندارند.
  در این صورت با حرکت کردن
  سوییپ‌لاین به هر نقطه که می‌رسیم، تنها نقاطی روشن هستند که
  در مشخصه
  $B$
  بزرگ‌تر هستند. حال در میان نقطه‌های روشن، بدنبال نقطه‌ای هستیم که
  $I_i > I_{now}$
  و
  $R_i > R_{now}$
  باشد. 

  خوش‌بختانه این مسئله نیز با استفاده از سگمنت قابل حل است.
  این بار هنگام روشن شدن یک نقطه، مقدار خانه
  $I_i$
  را برابر با
  $R_i$
  می‌کنیم.
  وقتی به نقطه‌ای می‌رسیم، 
  برای بررسی کردن اینکه افسردگی می‌گیرد یا نه،
  باید در میان نقاط روشنی که 
  $I$
  آن‌ها در بازه
  $[I_{now} + 1, \infty]$
  قرار دارد وجود یا عدم وجود نقطه‌ای که
  $R_i > R_{now}$
  باشد را بررسی کنیم.
  با گرفتن مقدار بیشینه بازه
  $[I_{now} + 1, \infty]$
  به خواسته خود می‌رسیم.

  در این مسئله نیز به کمک
  \LR{Sweepline}
  و
  \LR{Segment Tree}
  که درخواست‌های زیر را انجام می‌دهد، مسئله حل شد.
  حواستان به مساوی بودن مقادیر مشخصه‌ها باشد. 
  \begin{enumerate}
    \item
      مقدار خانه
      $idx$
      را به
      $value$
      تغییر بده.

    \item
      بیشینه بازه
      $L$
      تا
      $R$
      را بگو.
  \end{enumerate}

\end{problem}